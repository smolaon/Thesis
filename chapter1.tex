\pagenumbering{arabic}

\chapter{Úvod}

V dnešní době dochází v datových centrech k nasazování nových moderních technologii. Jednou z nich je například virtualizace v oblasti výpočetního výkonu a úložiště. Je již běžnou praxí, že v datových centrech vše běží na jedné fyzické infrastruktuře, na které existuje několik různých projektů zcela oddělených virtuálním prostředím. K vývoji došlo i v oblasti počítačových sítí. Díky softwarově definovaných sítím je možné vytvářet na sobě nezávislé sítě a vytvářet tak různé síťové topologie.  

Avšak i přes tyto nové technologie je dnes nejvíce síťové funkčnosti zatím soustředěno ve fyzických proprietárních zařízeních jako jsou routery, firewally či load balancery. To znamená, že provozovatelé počítačových sítí se při spouštění nových síťových služeb musí na tyto zařízení spoléhat. Což může vést k zdlouhavému nasazování, zvýšené spotřebě energii a investici do školení pracovníků pro dané proprietární zařízení. Tento fakt se snaží vyřešit právě virtualizované síťových funkcí, na kterou se zaměřuje tato diplomová práce.

Celá struktura této práce je rozdělena na 3 hlavní části. První dvě části jsou popisují oblast virtualizace síťových funkcí z teoretického hlediska a poslední pak z hlediska praktického. V první kapitole jsou vysvětleny hlavní pojmy a problematika této oblasti. Druhá je věnována popisu použitých technologii OpenStack a OpenContrail. V třetí části je následně ukázáno několik praktických příkladů. Na konci této práce dojde k závěrečnému shrnutí.

Závěrečná práce byla zpracována ve spolupráci s firmou tcp cloud a.s., která poskytuje implementace jednoho z nejlepších cloudových řešení na světě. Firma umožnila využít jejich stávající infrastrukturu v nejmodernějším datovém centru v České republice, které je v budově Technologického centra Písek s.r.o.



