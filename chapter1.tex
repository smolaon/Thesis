\pagenumbering{arabic}

\chapter{Úvod}

V dnešní době dochází v datových centrech k nasazování nových moderních technologii. V oblasti výpočetního výkonu a úložišť se jedná především o virtualizaci a cloud computing. Jak například udává \cite{Cloud_adoption} , tak v době psaní této práce 95\% IT profesionálů používá nějaký typ cloudové platformy. Přechází se tedy z hardwarově orientovaných data center na virtuální cloudová data centra. Je již tedy běžnou praxí, že v datových centrech vše běží na rozsáhlé fyzické infrastruktuře, která je abstrahovaná na jeden souvislé blok výpočetního výkonu a jeden souvislí blok úložiště.

Dalším takovýmto funkcionálním blokem, který je součástí datových centrech a je velice důžitou součástí infrastruktury velkých společností, jsou počítačové sítě. V oblasti počítačových sítích byl, oproti dvěma zmíněným oblastem, pomalejší vývoj inovací. Je to z důvodu toho, že počítačové sítě jsou velmi komplexní oblastí a také to, že produkční vývoj v telekomunikačním průmyslu se tradičně řídil přísnými standardy kvůli stabilitě a kvalitě komunikace, jak zmiňuje \cite{telco} . Přestože tento model v minulosti fungoval, tak vedl nevyhnutelně k dlouhým produkčním cyklům, pomalému tempu vývoje a spoléhání se na proprietární či specializovaný hardware.  

Avšak i zde je snaha změnit dosavadní návrh a fungování počítačových sítí. Je zde snaha využívat nové přístupy a technologie, které umožní flexibilní a rychlé nasazování nových síťových služeb a zárověň snížit jejich náklady. Jedná se především o využití již zmíněné virtualizace a programatické správy sítě. \cite{Toward_Unified}

Jedním z nových přístupů je virtualizace síťových funkcí (Network functions virtualization - NFV), kterou poprvé navrhl ETSI v publici \cite{NFV_paper2012} . Virtualizace síťových funkcí se zaměřuje na transformaci způsobu, jakým síťový architekti přistupují k oblasti počítačových sítí především u telekomunikačním poskytovatelů služeb. Snaha je tedy přesunout mnoho typů síťového příslušenství z fyzických síťových prvků do standardních průmyslově používaných serverů a úložišť, které mohou být umístěny v datových centrech či přímo u koncových zákazníků. Tímto lze dosáhnout virtuálních síťových funkcí, které mají naprosto stejnou funkcionalitu jako síťové funkce umístěné v síťových prvcích, avšak získávají výhody spojené s virtualizací a cloud computingem. NFV je relativně nová oblast, ve které je mnoho prostoru pro inovace a zlepšení, jak popisují \cite{NFVChalanges} a \cite{NFVState}. Jedním z nich je i management a orchestrace virtuálních síťových funkcí, kterou se zabývá tato práce.

Cílem této diplomové práce je analyzovat a navrhnout řešení pro management a orchestraci jednoduchých virtuálních síťových funkcí, které by mohli využít uživatelé cloudové platformy. Celé řešení spočívá v navržení architektury pro NFV frameworku, na kterém následně bude provedeno testovaní několika virtulálních síťových funkcí a jejich následné porovnání a zhodnocení. Výsledná řešení, které budou výstupem této práce, by měla sloužit jako obecný blueprint pro tvorbu NFV a VNF v cloudových prostředí. V celé práci budou využívány pouze aktuálně dostupné technologie, které by následně mohly být využity i v produkčním prostředí.

Celá struktura této práce je rozdělena na několik části. V druhé kapitole jsou vysvětleny hlavní pojmy a problematika oblasti virtualizace síťových funkcí. Třetí se zabývá refereční architekturou NFV a popisem možných technologii. Ve čtvrté kapitole je popsána již výsledná architektura a použité technologie společné s odůvodněních pro jejich vybrání. Pátá kapitola je následně věnována testování a realizaci jednotlivých virtuálních síťových funkcí. Na konci této práce je poté uvedeno závěrečné shrnutí.

Závěrečná práce byla zpracována ve spolupráci s firmou tcp cloud a.s., která poskytuje implementace jednoho z nejlepších cloudových řešení na světě. Firma umožnila využít jejich stávající infrastrukturu v nejmodernějším datovém centru v České republice, které je v budově Technologického centra Písek s.r.o.