\pagenumbering{arabic}

\chapter{Úvod}

V dnešní době dochází v datových centrech k nasazování nových moderních technologii. V oblasti výpočetního výkonu a úložišť se jedná především o virtualizaci a cloud computing. Jak například udává \cite{Cloud_adoption} , tak v době psaní této práce 95\% IT profesionálů používá nějaký typ cloudové platformy. Je již tedy běžnou praxí, že v datových centrech vše běží na jedné fyzické infrastruktuře, která je abstrahovaná na jeden souvislé blok výpočetního výkonu a jeden souvislí blok úložiště.

Dalším takovýmto funkcionálním blokem v datových centrech jsou počítačové sítě. V oblasti počítačových sítích byl, oproti dvěma zmíněným oblastem, pomalejší vývoj inovací. Je to z důvodu toho, že počítačové sítě jsou velmi komplexní oblastí a také to, že produkční vývoj v telekomunikačním průmyslu se tradičně řídil přísnými standardy kvůli stabilitě a kvalitě komunikace. Přestože tento model v minulosti fungoval, tak vedl nevyhnutelně k dlouhým produkčním cyklům, pomalému tempu vývoje a spoléhání se na proprietární či specializovaný hardware.

Avšak protože dochází k přechodu z hardwarově orientovaných data center na virtuální cloudová data centra. Je zde snaha využívat nové přístupy a technologie, které umožní flexibilní a rychlé nasazování nových síťových služeb a zárověň snížit jejich náklady.

Jedním z takovýchto nových přístupů je virtualizace síťových funkcí (Network functions virtualization - NFV). Virtualizace síťových funkcí se zaměřuje na transformaci způsobu, jakým síťový architekti přistupují k oblasti počítačových sítí a to pomocí stávájících a neustále se vyvíjejících virtualizačních technologii. Snaha je tedy přesunout mnoho typů síťového příslušenství z fyzických síťových prvků do standardních průmyslově používaných serverů a úložišť, které mohou být umístěny v datových centrech či přímo u koncových zákazníků. Tímto lze dosáhnout virtuálních síťových funkcí, které mají naprosto stejnou funkcionalitu jako síťové funkce umístěné v síťových prvcích, avšak získávají výhody spojené s virtualizací a cloud computingem.

Hlavním cílem této diplomové práce je navrhnout řešení pro jednoduché vytvoření virtuálních síťových funkcí, které by mohli využít uživatelé cloudové platformy. Současně by k jeho vytvoření měli být použity aktuálně dostupně technologie. Toto řešení musí být univerzální, nezávislé na vendorech a flexibilní.

//TODO doplnit až to bude hotové
Celá struktura této práce je rozdělena na několik části. V druhé kapitole jsou vysvětleny hlavní pojmy a problematika oblasti virtualizace síťových funkcí. Třetí je popisu návrhu řešení frameworku pro virtualizaci síťových funkcí a popisu použitých použitých technologii pro tento návrh. Ve čtvrté kapitole je věnována testování a ukázce, jak navržený framework funguje. Na konci této práce dojde k závěrečnému shrnutí.

Závěrečná práce byla zpracována ve spolupráci s firmou tcp cloud a.s., která poskytuje implementace jednoho z nejlepších cloudových řešení na světě. Firma umožnila využít jejich stávající infrastrukturu v nejmodernějším datovém centru v České republice, které je v budově Technologického centra Písek s.r.o.