\pagenumbering{arabic}

\chapter{Úvod}

V dnešní době dochází v datových centrech k nasazování nových moderních technologii. Jednou z nich je například virtualizace a to především v oblasti výpočetního výkonu a úložišť. Je již běžnou praxí, že v datových centrech vše běží na jedné fyzické infrastruktuře, která je abstrahovaná na jeden souvislé blok výpočetního výkonu a jeden souvislí blok úložiště. Dalším takovýmto funkcionálním blokem v datových centrech jsou počítačové sítě. Avšak v počítačových sítích byl, oproti dvěma zmíněným oblastem, pomalejší vývoj inovací a není zde tolik vyžívána virtualizace. Pro zvýšení efektivity je proto nutné, aby se počítačové sítě staly programovatelnými a mohli být spravovány z jednoho centrálního místa. 

Dnes je však zatím nejvíce síťové funkčnosti  soustředěno ve fyzických proprietárních zařízeních jako jsou routery, firewally či load balancery. To znamená, že provozovatelé počítačových sítí se při spouštění nových síťových služeb musí na tyto zařízení spoléhat. Což může vést k zdlouhavému nasazování, zvýšené spotřebě energii a investici do školení pracovníků pro dané proprietární zařízení. Zároveň zde není možnost, aby síť mohla být dynamicky ovládána dle aktuálních požadavků uživatelů sítě. Například vývojář nemůže hned nasadit aplikaci do produkce. Musí nejprve čekat na síťový tým než patřičně nakonfigurují síťové prvky pro správné a bezpečné fungování celé infrastruktury.

Virtualizace síťových funkcí se zaměřuje na transformaci způsobu, jakým síťový architekti přistupují k oblasti počítačových sítí a to pomocí stávájících a neustále se vyvíjejících virtualizačních technologii. Snaha je tedy přesunout mnoho typů síťového příslušenství z fyzických síťových prvků do standardních průmyslově používaných serverů a úložišť, které mohou být umístěny v datových centrech či přímo u koncových zákazníků. Tímto lze dosáhnout virtuálních síťových funkcí, které mají naprosto stejnou funkcionalitu jako síťové funkce umístěné v síťových prvcích.

Cílem této diplomové práce je analyzovat aktuální stav v oblasti virtualizace síťových funkcí. Dále je cílem navrhnout jednoduchý framework pro virtualizaci síťových funkcí spolu s ukázkou několik příkladů síťových funkcí. Celý framework by měl sloužit k možnosti rychlého a jednoduchého nasazení vybraných síťových funkcí. Současně by k jeho vytvoření měli být použity aktuálně dostupně technologie. Toto řešení musí být univerzální, nezávislé na vendorech a flexibilní.

Celá struktura této práce je rozdělena na několik části. V druhé kapitole jsou vysvětleny hlavní pojmy a problematika oblasti virtualizace síťových funkcí. Třetí je popisu návrhu řešení frameworku pro virtualizaci síťových funkcí a popisu použitých použitých technologii pro tento návrh. Ve čtvrté kapitole je věnována testování a ukázce, jak navržený framework funguje. Na konci této práce dojde k závěrečnému shrnutí.

Závěrečná práce byla zpracována ve spolupráci s firmou tcp cloud a.s., která poskytuje implementace jednoho z nejlepších cloudových řešení na světě. Firma umožnila využít jejich stávající infrastrukturu v nejmodernějším datovém centru v České republice, které je v budově Technologického centra Písek s.r.o.



