\pagenumbering{arabic}

\chapter{Úvod}

Z posledních 10 let došlo v IT k velkému rozšíření virtualizace, především v oblasti výpočetního výkonu a uložiště. Je již běžnou praxí, že v datových centrech existuje několik projektů, které běží na jednom hardwaru, ale logicky jsou zcela oddělené a nemohou se tedy ovlivňovat. V poslední době virtualizace dorazila i do oblasti počítačových sítí. Díky softwarově definovaných sítím je možné vytvářet na sobě nezávislé sítě a vytvářet tak různé síťové topologie.  Avšak i přes tuto novou technologii je v současné době nejvíce síťové funkčnosti zatím soustředěna ve fyzických zařízeních jako jsou routery, firewally či load balancery. Tento fakt se snaží vyřešit virtualizovat síťových funkcí.

Tato práce je rozdělena na 3 hlavní části. První dvě kapitoly popisují oblast virtualizace síťových funkcí z teoretického hlediska a poslední pak z hlediska praktického. V první kapitole jsou vysvětleny hlavní pojmy a problematika této oblasti. Druhá je věnována popisu použitých technologii OpenStack a OpenContrail. V třetí části je následně ukázáno několik praktických příkladů. Na konci této práce dojde k závěrečnému shrnutí.

Závěrečná práce byla vybrána ve spolupráci s firmou tcp cloud a.s., která poskytuje implementace jednoho z nejlepších cloudových řešení na světě. Firma umožnila využít jejich stávající infrastrukturu v nejmodernějším datovém centru v České republice, které je v budově Technologického centra Písek s.r.o.



