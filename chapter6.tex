\chapter{Závěr}

Virtualizace síťových funkcí je dnes jedno z aktuálních témat v oblasti počítačových sítí a IT světě. Je to z důvodů zvyšujících se požadavků na počítačové sítě, která je zapříčiněna rychlým vývojem a nasazováním nových technologii jako je např. cloud computing. Dalším důvodem jsou stoupající požadavky společností a telekomunikačních poskytovatelů síťových služeb na efektivnější a flexibilnější provoz počítačové sítě. 

Tato práce si dala za cíl prověřit a vyzkoušet možnosti managementu a orchestrace virtuálních síťových služeb. V práci byla zmíněna základní problematika NFV a zároveň bylo realizováno několik příkladů VNF, na kterým byly demonstrovány jejich aktuální možnosti automatizovaného managementu. Z výsledků práce lze usoudit, že aktuálně existují nástroje, které dokáží zajistit automatickou konfiguraci VNF, avšak je nutné zajistit řádnou podporu ze strany poskytovatelů těchto řešení.   

Jako vhodný předmětem pro další výzkum je možné označit integraci jednotlivých management nástrojů přímo cloudové platformy. Tím by byly zajištěny lepší možnosti automatizace. Dále je také možné označit za vhodný předmět k dalšímu výzkumu i podporu výběru fyzického umístění jednotlivých VNF. Tím by mohl být výrazně oblivněn jejich výkon.