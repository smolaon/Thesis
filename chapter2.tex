\chapter{Základní problematika virtualizace síťových funkcí}

Tato kapitola se zabývá základní analýzou a popisem problematiky spojené s oblastí virtuální síťových funkcí. Nejprve uveden přehled a krátký popis virtualizace a cloud computingu spolu s jejich přínosem pro IT. Následně je vysvětlena potřeba virtualizace síťových funkcí. Zde jsou porovnány jednotlivé přístupy k řešení a návrhu počítačových sítí. Zárověň je zde vysvětlen i základní koncept virtualizace síťových funkcí.

\section{Princip virtualizace}

Virtualizace je technologie, která změnila způsob, jakým se přistupuje k IT infrastruktuře. V tradičním modelu IT infrastuktury servery podporují pouze jeden operační systém v daném čase. Na tomto systému obvykle běží pouze jedna aplikace. Přestože by na tomto systému mohlo běžet více aplikací, tak je lepší držet aplikace odděleně na různých systech z důvodu minimalizace potencionálních bodů selhání. Pokud například nastane s aplikací problém, tak častým řešením je restartování systému. Pokud by na systému bylo více aplikací, znamenalo by to jejich vyřazení z provozu po dobu restartu, který může trvat velice dlouho. \cite{VM_book}

Z výše zmíněných důvodů došlo k velkému rozvoji a nasazování virtualizace. Virtualizací obecně označujeme techniky, které umožňují k dostupným hardwarovým zdrojům přistupovat jiným způsobem, než jakým fyzicky existují. Je tomu díky softwaru, který tento hardware abstrahuje a vytvoří tím virtuální prostředí. Virtualizované prostředí se dá snadněji přizpůsobit potřebám uživatelů, případně skrýt pro uživatele nepodstatné detaily (jako např. rozmístění hardwarových prostředků). Tím je tedy umožněno na jednom fyzickém serveru provozovat více od sebe oddělených virtuálních strojů, které mají každý svůj vlastní operačních systému s aplikacemi.

Software, který slouží pro virtualizaci, se nazývá hypervisor. Jak zmiňuje Popel a Goldberg v \cite{VM_architektura}, tak existují tyto dva základní typy hypervisorů:

\begin{itemize}
\item Typ 1 (Nativní - Bare-metal) - Tento hypervisor běží přímo na fyzickém hardwaru. Tím umožňuje provozovat více operačních systému na jednom fyzickém stroji. Příkladem takového hypervisoru je VMware ESXi, XEN a KVM.
\item Typ 2 (Hostovaný) - Na rozdíl od předchozího případu tento typ hypervisoru běží v prostředí operačního systému. Příkladem je například velice oblíbený Virtualbox či VMware Workstation.
\end{itemize}

Obrázek \ref{fig:virtualization} zobrazuje schématický popis obou typů hypervisorů a jejich rozdíl.

\begin{figure}[h]
\begin{centering}
\includegraphics[scale=0.5]{images/virtualization}
\par\end{centering}
\caption{Schéma hypervisorů \label{fig:virtualization}}
\end{figure}

Virtualizace je základní technologie, na které je postavena virtualizace síťových funkcí. Ve většině připadů je však využití pouze jednoho hypervizoru telekomunikačními provozovately a velkými společnostmi nedostačující a je proto nutné využít cloud computingu.

\section{Cloud Computing}

Existuje několik definic pro cloud computing. Nejčastější používaná definice pro cloud computing je \cite{nist_definition}, kde je uvedeno, že je to  model umožnující využívání společného poolu počítačových zdrojů vzdáleně přes počítačovou síť. Tyto zdroje mohou být flexibilně alokovány a uvolňovány dle potřeby. Základní charakteristiky cloudu jsou tedy:

\begin{itemize}
\item Služby dostupné na požádání
\item Všudypřítomný přístup k síti
\item Sdílení zdrojů
\item Vysokou elasticitu
\item Měření využitých zdrojů
\end{itemize}

Z technického hlediska je cloud resp. cloudoví infrastruktura několik společně propojených serverů v clusteru. Na těchto serverech běží hypervisor, který vytvoří virtuální infrastrukturu. Tímto způsobem je tvořen společný pool zdrojů. Pro vytváření cloudových služeb zde ještě musí existovat cloudová platforma, která dokáže celou tuto virtuální infrastrukturu centrálně spravovat.

V \cite{cloud_book} jsou uvedeny základní typy cloudu a cloudových služeb, které jsou dále popsány.

\subsection{Typy nasazení cloudových platforem}

Existuje několik základních modelů nasazení cloud computingu resp. cloudových platforem. Toto rozdělení je určeno způsobem jakým je cloud používám a poskytován. Tyto modely nasazení jsou následují:

\begin{itemize}
\item Privátní cloud - Privátní cloud je  infrastruktura  provozována  výhradně  v  rámci  jedné  organizace. Může být spravován interně nebo prostřednictvím třetí strany a hostování může být opět interní nebo externí.  Aby  mohl  podnik  využít  privátní  cloud,  musí  nejprve  navrhnout a uzpůsobit k tomuto účelu svoji stávající infrastrukturu, která musí být virtualizována.  Vlastní  přechod  vyvolává  řadu  bezpečnostních  otázek,  které  je třeba řešit, aby se zabránilo vážným zranitelnostem celého řešení. 
\item Veřejný cloud - Veřejné cloudové jsou cloudové služby, jako jsou aplikace, výpočetní výkon, úložiště a další, které jsou k dispozici široké  veřejnosti.  Služby  jsou  poskytovány  zdarma  nebo  podle modelu  platby  za  množství  použitých  služeb.  Je  zvykem,  že  veřejní  poskytovatelé cloudových  služeb,  jako  je  Amazon  AWS,  Microsoft  nebo  Google,  vlastní  a  provozují hardwarovou infrastrukturu a nabízejí k ní přístup pouze přes Internet. 
\item Hybridní cloud - Hybridní  cloud je  spojení  dvou  nebo  více  cloudů  (soukromých,  komunitních  nebo veřejných),  které  zůstávají  samostatné,  ale  jsou  těsně  propojeny.  Toto  složení  rozšiřuje možnosti  nasazení  cloudových  služeb  a  tím  umožňuje  IT  organizacím  využít  veřejné cloudové prostředky k uspokojení dočasných potřeb. Tato schopnost umožňuje hybridním cloudům škálovat přes více nezávislých cloudů. 
\item Komunitní cloud - V rámci komunitního cloudu sdílí infrastrukturu cloudu několik organizací, které mají společné zájmy (bezpečnost, dodržování předpisů, působnost, atd.). Komunitní cloud může být spravován interně nebo prostřednictvím třetí strany. Náklady jsou rozloženy mezi méně uživatelů než na veřejném cloudu. 
\end{itemize}

\subsection{Typy distribuce cloudových služeb}

Cloudové služby, které poskytují všechny výše zmíněné cloudové platformy, lze dále rozdělit do 3 základních kategorii na základě poskytované funkčnosti. 

\begin{itemize}
\item Infrastracture as a Service (IaaS) - Nejzákladnější  model  poskytování  cloudových  služeb. IaaS cloudové platformy nabízejí například výpočetní výkon, virtuální disky, blokové a souborové úložiště či virtuální sítě.  Poskytovatelé  IaaS  cloudových  platforem  poskytují tyto zdroje na vyžádání ze svých datových center. Toto je možné díky skupiny hypervisorů v rámci cloudu, které mohou provozovat velké množství virtuálních strojů  a  mají  schopnost  škálovat  poskytované  služby v  závislosti  na  měnících  se požadavcích přicházejících od zákazníků. Tento model může tedy sloužit i pro poskytnutí všech potřebných zdrojů celé infrastruktury pro virtualizaci síťových prvků, neboli Network Function Virtualization Infrastrakture as a Service.

\item Platform as a Service (PaaS) - V  modelu  Platforma  jako  služba  (PaaS)  hostují  poskytovatelé cloudových  služeb určitou počítačovou  platformu, kterou následně poskytují koncovým uživatelům přes Internet. Tato platforma většinou bývá prostředí nějakého operačního systému, prostředí  pro  běh  určitého programovacího  jazyka,  databáze  a  webový  server.  Vývojáři  aplikací tím pádem  mohou provozovat a případně vyvíjet svá softwarová řešení bez výrazných nákladů a složitého nákupu a konfiguraci potřebného hardwaru a softwaru. Některé PaaS platformy nastavuje výpočetní  a  úložné  prostředky  aplikace  automaticky  tak,  aby  odpovídala  aktuálním požadavkům aplikace bez nutnosti zásahu zákazníka. 

\item Software as a Servicec (SaaS) - V modelu SaaS provozují poskytovatelé cloudových služeb aplikační software v cloudu a uživatelé  k  tomuto  softwaru  přistupují  pomocí  klientského  software  (např.  webové prohlížeče). Uživatelé  cloudu tedy  nespravují infrastrukturu  ani platformu, kde aplikace běží.  Není  proto  třeba zde nic instalovat  a  spouštět  aplikace  na  vlastních  počítačích uživatele, což velmi zjednodušuje údržbu. Cloudové aplikace se liší od ostatních aplikací v možnostech  škálování, kterého  může  být  dosaženo  díky  distribuci  úkolů  na  více virtuálních strojů, a tím reagovat na měnící se poptávku. Tento proces je pro uživatele služby transparentní, uživatel vidí pouze jeden přístupový bod pro danou aplikaci. 
\end{itemize}

\section{Tradiční počítačové sítě}

Pohledem na tradiční počítačovou sít zjistíme, že nejvíce síťové funkčnosti je soustředěno ve fyzických proprietárních zařízeních jako jsou routery, firewally či load balancery. To znamená, že provozovatelé počítačových sítí se při spouštění nových síťových služeb musí na tyto zařízení spoléhat. Což může vést k zdlouhavému nasazování, zvýšené spotřebě energii a investici do školení pracovníků pro dané proprietární zařízení. Zároveň zde není možnost, aby síť mohla být dynamicky ovládána dle aktuálních požadavků uživatelů sítě. Například vývojář nemůže hned nasadit aplikaci do produkce. Musí nejprve čekat na síťový tým než patřičně nakonfigurují síťové prvky pro správné a bezpečné fungování celé infrastruktury.

\section{Softwarově definované sítě - SDN}

Softwarově definované sítě (SDN) je jednou z nových technologii, která se snaží vylepšit a automatizovat správu stávajících počítačových sítí. Dle \cite{SDN_clanek} jde o koncept, ve které je oddělena řídící logika (control plane) z jednotlivých routerů a switchů, které přeposílají traffic (data plane). Tím, že dojde k oddělení datové a řídící vrstvy, se routery a switche stanou pouze přeposílající data a veškerá řídí logika může být implementována v jednom logicky centrálním místě (SDN Controller). Z tohoto centrálního místa lze do jednotlivých routerů a switchů předávat instrukce pomocí aplikačních programovacích rozhraní (API). Samotný SDN Controller také obsahuje API, které mohou využívat aplikace a tím řídit, resp. programovat celou počítačovou síť.

\begin{figure}[h]
\begin{centering}
\includegraphics[scale=0.60]{images/SDN}
\par\end{centering}
\caption{Schéma SDN, převzato z \cite{SDN_clanek}\label{fig:SDN}}
\end{figure}

Obrázek č. \ref{fig:SDN} úkazuje jednoduché schéma softwarově definovaných sítí. Celou architekturu lze tedy rozdělit do 3 logických vrstev, které spolu komunikují pomocí API. 

\begin{itemize}
\item Aplikační vrstva - Na této úrovni se nachází samotné síťové aplikace jako jsou například DHCP, ACL, NAT, DNS a další. Jejich vytváření by
mělo být poskytováno prostřednictvím nižší vrstvy, nazývané northbound API.
\item Northbound APIs - Toto API využívají aplikace pro komunikaci s SDN controllerem. 
\item Control vrstva - V této vrstvě je centralizována veškerá logika, které dříve byla v síťových prvcích.
\item Southbound APIs - Jedná se o skupinu API protokolů, které pracují mezi vrstvou infrastruktury a control vrstvou. Jejím hlavním úkolem je komunikace, která povoluje SDN controleru instalovat na samotné síťové prvky rozhodnutí definované v aplikační vrstvě.
\item Vrstva infrastruktura - Nejnižší vrstvou je samotný hardware pro předávání datagramů na fyzické úrovni. Pro funkčnost celé architektury je nutné, aby zde byla nasazena zařízení, která umí přijímat pokyny od control plane skrze southbound API.

\end{itemize}

Přestože Softwarově definované sítě a virtualizace síťových funkcí jsou dvě různé technologie a koncepty, tak se navzájem se doplňují. Fakt, že SDN umožňuje programaticky ovládat počítačovou síť, lze využít pro poskytnutí programovatelné konektivity mezi jednotlivými virtuálními síťovými funkcemi. Naopak SDN může využít NFV tím, že implementuje potřebné síťové funkce jako software. Může tak virtualizovat SDN Controller, který tak může běžet na co nejvhodnějším místě v datovém centru. Je vidět, že tyto dvě technologie se dobře doplňují, proto jsou často součástí jednoho řešení. \cite{SDN_book}


\section{Virtualizované síťové funkce - NFV}

//TODO Obrazek + popis jak je to dobry
//SW model

//TODO Obrazek pro Home
//TODO Obrazek pro Cloud
//TODO Obrazek pro Telco

Hlavní cíle tohoto řešení jsou zlepšit následující aspekty provozu telekomunikačních sítí:

\begin{itemize}
\item Smíření investičních nákladů – snížení potřeby nákupu jednoúčelových hardwarových zařízení, možnost platby pouze za využité kapacity a snížení rizik přílišného předimenzování kapacit
\item Snížení provozních nákladů – snížení prostoru, napájení a požadavky na chlazení, zjednodušení správy a řízení síťových služeb
\item Urychlení Time-to-market – zkrácení doby pro nasazení nových síťových služeb, chopení se nových příležitosti na trhu, vyhovění potřebám zákazníka
\item Doručit agilitu a flexibilitu – možnost rychle škálovat (rozšiřovat nebo zmenšovat služby) dle měnících se požadavků od zákazníka. Podpora služeb, které mají být dodány pomocí softwaru na libovolném standardním serverovém hardwaru
\end{itemize}

Jak je uvedeno v \cite{NFVState} a \cite{NFVChalanges}, tak celá myšlenka je založena na tom, že dojde k separování softwarové funkcionality v síťových prvcích od proprietárního hardwaru, na kterém běží. To umožní se síťovými funkcemi zacházek jako s klasickými softwarovými aplikacemi, které mohou běžet na standardním komerčně dostupných serverech jenž organizace v současnosti používají. Tím bude zároveň umožněno flexibilní nasazování těchto síťových funkcí a jejich dynamický provisioning. Díky tomu, že jsou síťová funkce odděleny od hardwaru, tak je také možné jejich vhodnější umístění v topologii. To znamená dle požadavků na umístění mohou být nasazeny v datových centrech, síťových uzlech či přímo v uživatelově koncovém bodě. Hlavní koncept virtualizace síťových funkcí znázorňuje obrázek č. \ref{fig:vize_NFV}. 

\begin{figure}[h]
\begin{centering}
\includegraphics[scale=0.5]{images/vize_NFV}
\par\end{centering}
\caption{Koncept virtualizace síťových funkcí (NFV)\label{fig:vize_NFV}}
\end{figure}

Za zmínění stojí poznámka v \cite{NFVState}, kde je řečeno, že obecný koncept oddělení síťové funkce od hardwaru ještě nutně neznamená potřebu využití virtualizace. Protože budou síťové funkce dostupně jako software, tak mohou být nainstalovány a provozovány přímo na fyzickém stroji. Ovšem rozdíl je, že tento stroj již nebude speciální hardware, ale klasický server. Tento scénář může být do jisté míry použit při nasazovaní síťových funkcí v malém měřítku např. v uživatelských koncových bodech. Avšak pro plné využití všech výše zmíněných výhod, které jsou třeba ve velkých datových centrech, je třeba s použitím virtualizace počítat. To vše umocňuje fakt, že většina datových center v současnosti již využívá cloud computing. 

\section{Souvislost SDN a NFV}

\begin{figure}[h]
\begin{centering}
\includegraphics[scale=0.5]{images/sdn_nfv}
\par\end{centering}
\caption{ (NFV)\label{fig:vize_NFV}}
\end{figure}

//Tabulka rozdílů porovnani

\cite{Toward_NFV}

\section{Service Chaining} \label{sub:SDN}

Jednou z výhod NFV je možnost využít Service Chaining. Service chaining je ve skutečnosti součást SDN. Jde o princip jakým lze dynamicky pospojovat jednotlivé VNF a ovladat tak toky v síti. \cite{SDN_book}

Service chaining není ve skutečnosti nic nového. V klasických počítačových sítích je používán také, ale pomocí fyzických síťových prvků. Jedná se zjednodušeně o způsob zapojení mezi jednotlivými síťovými prvky (či VNF) a způsob, jakým na sebe navazují. Příklad takového zapojení je vidět na obrázku č. \ref{fig:service_chaining}. Zde se provozovatel sítě rozhodl, že odchozí data z klientských stanic musí jít přes firewall, IDS a nakonec přes NAT do Internetu. Příchozí data mají logicky obrácené pořadí. Toto zapojení funguje dobře pro síť, kde není třeba rozlišovat cestu jakou proudí data jednotlivých uživatelských stanic. Ale není to optimální řešení pro sítě s více uživateli, kde každý požaduje jinou síťovou funkci. Potřeba jednotlivých síťových služeb se samozřejmě může v čase měnit. Příklad takové sítě lze nalézt ve většině datových center. 

\begin{figure}[h]
\begin{centering}
\includegraphics[scale=0.55]{images/service_chaining}
\par\end{centering}
\caption{Ukázka klasického service chainigu pomocí fyzických síťových prvků\label{fig:service_chaining}}
\end{figure}

Zde tedy přichází na řadu VNF spolu s SDN. Protože jednotlivé VNF existují jako virtuální stroje, tak mohou být dynamicky nasazovány dle aktuálním požadavků jednotlivých klientů a pomocí SDN mohou být tyto VM dynamicky pospojovány. Obrázek č. \ref{fig:service_chaining_new} ukazuje schéma zapojení, kde každý klient může mít jinou požadovanou cestu do internetu. Je možná i varianta, kde každý klient má své vlastní VNF s jinou konfigurací.

\begin{figure}[h]
\begin{centering}
\includegraphics[scale=0.55]{images/service_chaining_new}
\par\end{centering}
\caption{Ukázka VNF service chainigu\label{fig:service_chaining_new}}
\end{figure}